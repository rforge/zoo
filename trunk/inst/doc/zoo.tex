\documentclass{Z}
%% need no \usepackage{Sweave}

\author{Achim Zeileis\\Wirtschaftsuniversit\"at Wien \And
        Gabor Grothendieck}
\Plainauthor{Achim Zeileis, Gabor Grothendieck}

\title{\pkg{zoo}: An \proglang{S3} Class and Methods for
  Indexed Totally Ordered Observations}
\Plaintitle{zoo: An S3 Class and Methods for
  Indexed Totally Ordered Observations}
%% \Shorttitle{\pkg{zoo}: \proglang{S3} Infrastructure for Indexed Totally Ordered Observations}

\Keywords{totally ordered observations, irregular time series,
  \proglang{S3}, \proglang{R}}
\Plainkeywords{totally ordered observations, irregular time series, S3, R}

\Abstract{
  \pkg{zoo} is an \proglang{R} package providing an \proglang{S3}
  class with methods for indexed totally ordered observations, such as
  irregular time series. Its key design goals are independence of a
  particular index/time/date class and consistency with base
  \proglang{R} and the \code{"ts"} class for
  regular time series. This paper describes how these are achieved
  within \pkg{zoo} and provides several illustrations 
  of the available methods for \code{"zoo"} objects which include
  plotting, merging and binding, several mathematical operations,
  extracting and replacing value and index, coercion and \code{NA}
  handling.
}

\begin{document}


%\VignetteIndexEntry{zoo: An S3 Class and Methods for Indexed Totally Ordered Observations}
%\VignetteDepends{zoo,tseries,fBasics,strucchange}
%\VignetteKeywords{totally ordered observations, irregular time series, S3, R}
%\VignettePackage{zoo}



%% TODO:
%%
%% the zoo package seems disarmingly simple so I think it might be a good
%% idea to give some measure of size in the abstract and introduction such 
%% as 50+ functions and methods since many might not realize that it has 
%% that much functionality.
%%
%% I think we need both [ and also [<- and not sure if subsetting covers
%% both or not.
%% 
%% simplify random indexes
%%
%% check cbind.zoo naming

\section{Introduction} \label{sec:intro}

The \proglang{R} system for statistical computing
\citep[\url{http://www.R-project.org/}]{zoo:R:2004}
ships with a a class for regularly spaced time series,
\code{"ts"} in package \pkg{stats}, but has no native class for
irregularly spaced time series. With the increased interest in
computational finance with \proglang{R} over the last years
several implementations of classes for irregular time series 
emerged which are aimed particularly at finance applications.
These include the \proglang{S3} classes \code{"timeSeries"}
in package \pkg{fBasics} from the \pkg{Rmetrics} bundle \citep{zoo:fBasics:2004}
and \code{"irts"} in package \pkg{tseries} \citep{zoo:tseries:2004}
and the \proglang{S4} class \code{"its"} in package \pkg{its} \citep{zoo:its:2004}.
With these packages available, why would anybody want yet another 
package providing infrastructure for irregular time series?
The above mentioned implementations have in common that they are restricted to a particular
class for the time scale: the former implementation comes with its own time class
\code{"timeDate"} built on top of the \code{"POSIXt"} classes
available in base \proglang{R} whereas the latter two use \code{"POSIXct"} directly.
And this was the starting point for the \pkg{zoo} project: the first author
of the present paper needed
more general support for ordered observations, independent of a particular
index class, for the package \pkg{strucchange}
\citep{zoo:Zeileis+Leisch+Hornik:2002}. Hence the package was called
\pkg{zoo} which stands for \underline{Z}'s \underline{o}rdered \underline{o}bservations.
Since the first release, a major part of the additions to \pkg{zoo}
were provided by the second author of this paper, so that the name
of the package does not really reflect the authorship anymore.
Nevertheless, independence of a particular index class remained one
the most important design goal. While the package evolved to its current
status, a second key design goal became more and more clear: to provide
methods to standard generic functions for the \code{"zoo"} class that 
are similar to those for the \code{"ts"} class (and base \proglang{R} in
general) such that the usage of \pkg{zoo} is rather intuitive because
only few additional commands have to be learned.

This paper describes how these design goals are implemented in \pkg{zoo}.
Section~\ref{sec:zoo-class} explains how \code{"zoo"} objects are created
and illustrates how the corresponding methods for plotting, merging and
binding, several mathematical operations, extracting and replacing value
and index, coercion and \code{NA} handling can be used. Section~\ref{sec:combining}
outlines how other packages can build on this basic infrastructure
before Section~\ref{sec:summary} gives a few summarizing remarks.

\clearpage

\section[The class "zoo" and its methods]{The class \code{"zoo"} and its methods}
\label{sec:zoo-class}


\subsection[Creation of "zoo" objects]{Creation of \code{"zoo"} objects}
\label{sec:zoo}

The simple idea for the creation of \code{"zoo"} objects is to have
some vector or matrix of observations \code{x} which are totally ordered
by some index vector. In time series applications this index is measure of
time but every other numeric, character or even more abstract vector that
provides a total ordering of the observations is also suitable. Objects
of class \code{"zoo"} are created by the function
\begin{Scode}
zoo(x, order.by)
\end{Scode}
where \code{x} is the vector or matrix of observations and \code{order.by}
is the index by which the observations should be ordered. It has to be
of the same length as \code{NROW(x)}, i.e., either the same length as \code{x}
for vectors or the same number of rows for matrices. The \code{"zoo"} object
created is essentially the vector/matrix as before but has an additional
\code{"index"} attribute in which the index is stored. Both the value \code{x}
and the index can, in principle, be of arbitrary classes. However, most of the
following methods (plotting, aggregating, mathematical operations) for \code{"zoo"}
objects are typically only useful for numeric values \code{x}. In contrast, special
effort in the design was put into independence from a particular class for
the index vector. In \pkg{zoo}, it is assumed that combination \code{c()},
querying the \code{length()}, value matching \code{MATCH()}, subsetting \code{[,},
and, of course, ordering \code{ORDER()} work when applied to the index. 
This is the case, e.g., for standard numeric and character vectors and for
vectors of classes \code{"Date"}, \code{"POSIXct"} or \code{"times"}
from package \pkg{chron}, but not for the class \code{"dateTime"} in \pkg{fBasics}.
In the latter case, the solution is to provide methods for the above mentioned
functions so that indexing \code{"zoo"} objects with \code{"dateTime"} vectors works.
To achieve this  independence of the index class, new generic functions for
ordering (\code{ORDER()}) and value matching (\code{match()}) are introduced
as the corresponding base functions \code{order()} and \code{match()} are 
non-generic. The default methods simply call the corresponding base functions, i.e.,
no new method needs to be introduced for a particular index class if the 
non-generic functions \code{order()} and \code{match()} work for this class.

To illustrate the usage of \code{zoo}, we first load the package and set the
random seed to make the examples in this paper exactly reproducible.

\begin{Schunk}
\begin{Sinput}
> library(zoo)
> set.seed(1071)
\end{Sinput}
\end{Schunk}

Then, we create two vectors \code{z1} and \code{z2} with \code{"POSIXct"} 
indexes, one with random values
\begin{Schunk}
\begin{Sinput}
> z1.index <- ISOdatetime(2004, rep(1:2, 5), sample(28, 10), 0, 
+     0, 0)
> z1.value <- rnorm(10)
> z1 <- zoo(z1.value, z1.index)
\end{Sinput}
\end{Schunk}
and one with a sinus wave
\begin{Schunk}
\begin{Sinput}
> z2.index <- as.POSIXct(paste("2004-", rep(1:2, 5), "-", sample(1:28, 
+     10), sep = ""))
> z2.value <- sin(2 * 1:10/pi)
> z2 <- zoo(z2.value, z2.index)
\end{Sinput}
\end{Schunk}
Furthermore, we create a matrix \code{Z} with random values and a \code{"Date"}
index
\begin{Schunk}
\begin{Sinput}
> Z.index <- structure(sample(12450:12500, 10), class = "Date")
> Z.value <- matrix(rnorm(30), ncol = 3)
> colnames(Z.value) <- c("Aa", "Bb", "Cc")
> Z <- zoo(Z.value, Z.index)
\end{Sinput}
\end{Schunk}
Note, that in the above examples the creation of indexes might seem a bit awkward
at first sight, but this is only an artefact of the need for random generation
of random dates for this illustration. In ``real world'' applications, the indexes
are typically part of the raw data set read into \proglang{R}. See Section~\ref{sec:combining}
for such examples.

Methods to several standard generic functions are available for
\code{"zoo"} objects, such as \code{print}, \code{summary}, \code{str}, \code{head},
\code{tail} and \code{[} (subsetting), a few of which are illustrated in
the following. 

There are three printing code styles for \code{"zoo"} objects: vectors are default
printed in \code{"horizontal"} style
\begin{Schunk}
\begin{Sinput}
> z1
\end{Sinput}
\begin{Soutput}
Value:
 [1]  0.74675994  0.02107873 -0.29823529  0.68625772  1.94078850  1.27384445
 [7]  0.22170438 -2.07607585 -1.78439244 -0.19533304

Index:
 [1] "2004-01-05 Eastern Standard Time" "2004-01-14 Eastern Standard Time"
 [3] "2004-01-19 Eastern Standard Time" "2004-01-25 Eastern Standard Time"
 [5] "2004-01-27 Eastern Standard Time" "2004-02-07 Eastern Standard Time"
 [7] "2004-02-12 Eastern Standard Time" "2004-02-16 Eastern Standard Time"
 [9] "2004-02-20 Eastern Standard Time" "2004-02-24 Eastern Standard Time"
\end{Soutput}
\begin{Sinput}
> z1[3:7]
\end{Sinput}
\begin{Soutput}
Value:
[1] -0.2982353  0.6862577  1.9407885  1.2738445  0.2217044

Index:
[1] "2004-01-19 Eastern Standard Time" "2004-01-25 Eastern Standard Time"
[3] "2004-01-27 Eastern Standard Time" "2004-02-07 Eastern Standard Time"
[5] "2004-02-12 Eastern Standard Time"
\end{Soutput}
\end{Schunk}
and matrices in \code{"vertical"} style
\begin{Schunk}
\begin{Sinput}
> Z
\end{Sinput}
\begin{Soutput}
           Aa          Bb          Cc         
2004-02-02  1.25543390  0.68157316 -0.63292049
2004-02-08 -1.49458326  1.32341223 -1.49442269
2004-02-09 -1.87462247 -0.87329289  0.62733971
2004-02-21 -0.14538608  0.45234903 -0.14597401
2004-02-22  0.22542418  0.53838938  0.23136133
2004-02-29  1.20695518  0.31814222 -0.01129202
2004-03-05 -1.20861025  1.42379785 -0.81614483
2004-03-10 -0.11039563  1.34774254  0.95522468
2004-03-14  0.84202385 -2.73842019  0.23150695
2004-03-20 -0.19019104  0.12308872 -1.51862157
\end{Soutput}
\begin{Sinput}
> Z[1:3, 2:3]
\end{Sinput}
\begin{Soutput}
           Bb         Cc        
2004-02-02  0.6815732 -0.6329205
2004-02-08  1.3234122 -1.4944227
2004-02-09 -0.8732929  0.6273397
\end{Soutput}
\end{Schunk}
Additionally, there is a \code{"plain"} style which simply first prints the value 
and then the index.

Summaries and most other methods for \code{"zoo"} objects are carried out
column wise, reflecting the rectangular structure indexed by rows. In addition,
a summary of the index is provided.

\begin{Schunk}
\begin{Sinput}
> summary(z1)
\end{Sinput}
\begin{Soutput}
     Index                           z1          
 Min.   :2004-01-05 00:00:00   Min.   :-2.07608  
 1st Qu.:2004-01-20 12:00:00   1st Qu.:-0.27251  
 Median :2004-02-01 12:00:00   Median : 0.12139  
 Mean   :2004-02-01 09:36:00   Mean   : 0.05364  
 3rd Qu.:2004-02-15 00:00:00   3rd Qu.: 0.73163  
 Max.   :2004-02-24 00:00:00   Max.   : 1.94079  
\end{Soutput}
\begin{Sinput}
> summary(Z)
\end{Sinput}
\begin{Soutput}
     Index                  Aa                Bb                Cc          
 Min.   :2004-02-02   Min.   :-1.8746   Min.   :-2.7384   Min.   :-1.51862  
 1st Qu.:2004-02-12   1st Qu.:-0.9540   1st Qu.: 0.1719   1st Qu.:-0.77034  
 Median :2004-02-25   Median :-0.1279   Median : 0.4954   Median :-0.07863  
 Mean   :2004-02-25   Mean   :-0.1494   Mean   : 0.2597   Mean   :-0.25739  
 3rd Qu.:2004-03-08   3rd Qu.: 0.6879   3rd Qu.: 1.1630   3rd Qu.: 0.23147  
 Max.   :2004-03-20   Max.   : 1.2554   Max.   : 1.4238   Max.   : 0.95522  
\end{Soutput}
\end{Schunk}

 
\subsection{Plotting}
\label{sec:plot}

The \code{plot} method for \code{"zoo"} objects, in particular for
multivariate \code{"zoo"} series, is based on the corresponding
method for (multivariate) regular time series. It relies on \code{plot}
and \code{lines} methods being available for the index class which can
plot the index against the observations.

By default the \code{plot} method creates a panel for each series
\begin{Schunk}
\begin{Sinput}
> plot(Z)
\end{Sinput}
\end{Schunk}
but can also display all series in a single panel
\begin{Schunk}
\begin{Sinput}
> plot(Z, plot.type = "single", col = 2:4)
\end{Sinput}
\end{Schunk}
where in both cases additional graphical parameters like color \code{col},
plotting character \code{pch} and line type \code{lty} can be
expanded to the number of series. But the \code{plot} method for
\code{"zoo"} objects offers some more flexibility in specification
of graphical parameters as in
\begin{Schunk}
\begin{Sinput}
> plot(Z, type = "b", lty = 1:3, pch = list(Aa = 1:5, Bb = 2, Cc = 4), 
+     col = list(Bb = 2, 4))
\end{Sinput}
\end{Schunk}
The argument \code{lty} behaves as before and sets every series in another
line type. The \code{pch} argument is a named list that assigns to each series
a different vector of plotting characters each of which is expanded to the 
number of observations. Such a list does not necessarily have to include the names of all
series, but can also specify a subset. For the remaining series the default parameter
is then used which can again be changed: e.g., in the above example series \code{"Bb"} is
plotted in red and all remaining series in blue. The results of the multiple panel plots
are depicted in Figure~\ref{fig:plot13} and the single panel plot in \ref{fig:plot2}.

\begin{figure}[tbh]
\begin{center}
\includegraphics{zoo-plot2-repeat}
\caption{\label{fig:plot2} Example of a single panel plot}
\end{center}
\end{figure}


\begin{figure}[p]
\begin{center}
\includegraphics{zoo-plot1-repeat}
\includegraphics{zoo-plot3-repeat}
\caption{\label{fig:plot13} Examples of multiple panel plots}
\end{center}
\end{figure}


\subsection{Merging and binding}
\label{sec:merge}

As for many rectangular data formats in \proglang{R}, there are
both methods for combining the rows and columns of \code{"zoo"}
objects respectively. For the \code{rbind} method the number of
columns of the combined objects has to be identical and the
indexes may not overlap.
\begin{Schunk}
\begin{Sinput}
> rbind(z1[5:10], z1[2:3])
\end{Sinput}
\begin{Soutput}
Value:
[1]  0.02107873 -0.29823529  1.94078850  1.27384445  0.22170438 -2.07607585
[7] -1.78439244 -0.19533304

Index:
[1] "2004-01-14 Eastern Standard Time" "2004-01-19 Eastern Standard Time"
[3] "2004-01-27 Eastern Standard Time" "2004-02-07 Eastern Standard Time"
[5] "2004-02-12 Eastern Standard Time" "2004-02-16 Eastern Standard Time"
[7] "2004-02-20 Eastern Standard Time" "2004-02-24 Eastern Standard Time"
\end{Soutput}
\end{Schunk}
The \code{cbind} method by default combines the columns by the union of
the indexes and fills the created gaps by \code{NA}s.
\begin{Schunk}
\begin{Sinput}
> cbind(z1, z2)
\end{Sinput}
\begin{Soutput}
           ..1         ..2        
2004-01-03          NA  0.94306673
2004-01-05  0.74675994 -0.04149429
2004-01-14  0.02107873          NA
2004-01-17          NA  0.59448077
2004-01-19 -0.29823529 -0.52575918
2004-01-24          NA -0.96739776
2004-01-25  0.68625772          NA
2004-01-27  1.94078850          NA
2004-02-07  1.27384445          NA
2004-02-08          NA  0.95605566
2004-02-12  0.22170438 -0.62733473
2004-02-13          NA -0.92845336
2004-02-16 -2.07607585          NA
2004-02-20 -1.78439244          NA
2004-02-24 -0.19533304          NA
2004-02-25          NA  0.56060280
2004-02-26          NA  0.08291711
\end{Soutput}
\end{Schunk}
In fact, the \code{cbind} method is synonymous to the \code{merge}
method which also allows for combining the columns by the intersection
of the indexes using the argument \code{all = FALSE}.
\begin{Schunk}
\begin{Sinput}
> merge(z1, z2, all = FALSE)
\end{Sinput}
\begin{Soutput}
           z1          z2         
2004-01-05  0.74675994 -0.04149429
2004-01-19 -0.29823529 -0.52575918
2004-02-12  0.22170438 -0.62733473
\end{Soutput}
\end{Schunk}
Additionally, the filling pattern can be changed and the naming of the
columns can be modified. In the case of merging of objects with 
different index classes, \proglang{R} gives a warning and tries to
coerce the indexes, but this is generally rather difficult


Another function which performs operations along a subset of indexes
is \code{aggregate}, which is therefore discussed in this section although
it does not combine several objects. Using the \code{aggregate} method, \code{"zoo"} objects
are split into subsets along a coarser index grid,
summary statistics are computed for each and then the 
reduced object is returned. In the following example,
first a function is set up which returns for a given \code{"Date"}
value the corresponding first of the month. This function is then
used to compute the coarser grid for the \code{aggregate} call: in
the first example the mean of the observations in the month
is returned, in the second example the last observation.
%%FIXME: maybe also firstofquarter, see man pages

\begin{Schunk}
\begin{Sinput}
> firstofmonth <- function(x) as.Date(sub("..$", "01", format(x)))
> aggregate(Z, firstofmonth(Z.index), mean)
\end{Sinput}
\begin{Soutput}
           Aa          Bb          Cc         
2004-02-01  0.53820841  0.04508597 -0.12412352
2004-03-01 -1.18080051  0.58156655 -0.45730045
\end{Soutput}
\begin{Sinput}
> aggregate(Z, firstofmonth(Z.index), tail, 1)
\end{Sinput}
\begin{Soutput}
           Aa         Bb         Cc        
2004-02-01 -0.1901910  0.1230887 -1.5186216
2004-03-01 -1.2086102  1.4237978 -0.8161448
\end{Soutput}
\end{Schunk}


\subsection{Mathematical operations}
\label{sec:Ops}

To allow for standard mathematical operations among \code{"zoo"}
objects, \pkg{zoo} extends group generic functions \code{Ops}.
These perform the operations only for the intersection of the
indexes of the objects. Hence, the summation of \code{z1} and
\code{z2} yields
\begin{Schunk}
\begin{Sinput}
> z1 + z2
\end{Sinput}
\begin{Soutput}
Value:
[1]  0.7052657 -0.8239945 -0.4056304

Index:
[1] "2004-01-05 Eastern Standard Time" "2004-01-19 Eastern Standard Time"
[3] "2004-02-12 Eastern Standard Time"
\end{Soutput}
\end{Schunk}

Additionally, methods for transposing \code{t} of \code{"zoo"}
objects---which coerces to a matrix before, see below---and 
computing cumulative quantities such as
\code{cumsum}, \code{cumprod}, \code{cummin}, \code{cummax}
which are all applied column wise.
\begin{Schunk}
\begin{Sinput}
> cumsum(Z)
\end{Sinput}
\begin{Soutput}
           Aa         Bb         Cc        
2004-02-02  1.2554339  0.6815732 -0.6329205
2004-02-08 -0.2391494  2.0049854 -2.1273432
2004-02-09 -2.1137718  1.1316925 -1.5000035
2004-02-21 -2.2591579  1.5840415 -1.6459775
2004-02-22 -2.0337337  2.1224309 -1.4146162
2004-02-29 -0.8267785  2.4405731 -1.4259082
2004-03-05 -2.0353888  3.8643710 -2.2420530
2004-03-10 -2.1457844  5.2121135 -1.2868283
2004-03-14 -1.3037606  2.4736933 -1.0553214
2004-03-20 -1.4939516  2.5967820 -2.5739429
\end{Soutput}
\end{Schunk}


\subsection{Extracting and replacing the value and the index}
\label{sec:window}

\pkg{zoo} provides several generic functions and methods
to work on the value (or data) contained in a \code{"zoo"} object, the
index (or time) attribute associated to it, and on both data and
index.

The value stored in \code{"zoo"} objects can be extracted by
\code{value} which strips off all \code{"zoo"}-specific attributes and 
it can be replaced using \code{value<-}. Both are new generic functions
with methods for \code{"zoo"} objects as illustrated in the following
example.
\begin{Schunk}
\begin{Sinput}
> value(z1)
\end{Sinput}
\begin{Soutput}
 [1]  0.74675994  0.02107873 -0.29823529  0.68625772  1.94078850  1.27384445
 [7]  0.22170438 -2.07607585 -1.78439244 -0.19533304
\end{Soutput}
\begin{Sinput}
> value(z1) <- 1:10
> z1
\end{Sinput}
\begin{Soutput}
Value:
 [1]  1  2  3  4  5  6  7  8  9 10

Index:
 [1] "2004-01-05 Eastern Standard Time" "2004-01-14 Eastern Standard Time"
 [3] "2004-01-19 Eastern Standard Time" "2004-01-25 Eastern Standard Time"
 [5] "2004-01-27 Eastern Standard Time" "2004-02-07 Eastern Standard Time"
 [7] "2004-02-12 Eastern Standard Time" "2004-02-16 Eastern Standard Time"
 [9] "2004-02-20 Eastern Standard Time" "2004-02-24 Eastern Standard Time"
\end{Soutput}
\end{Schunk}

The index associated with a \code{"zoo"} object can be extracted
by \code{index} and modified by \code{index<-}. As the interpretation
of the index as ``time'' in time series applications is more natural,
there are also synonymous methods \code{time} and \code{time<-}. 
Hence, the following two commands return equivalent results
\begin{Schunk}
\begin{Sinput}
> index(z2)
\end{Sinput}
\begin{Soutput}
 [1] "2004-01-03 Eastern Standard Time" "2004-01-05 Eastern Standard Time"
 [3] "2004-01-17 Eastern Standard Time" "2004-01-19 Eastern Standard Time"
 [5] "2004-01-24 Eastern Standard Time" "2004-02-08 Eastern Standard Time"
 [7] "2004-02-12 Eastern Standard Time" "2004-02-13 Eastern Standard Time"
 [9] "2004-02-25 Eastern Standard Time" "2004-02-26 Eastern Standard Time"
\end{Soutput}
\begin{Sinput}
> time(z2)
\end{Sinput}
\begin{Soutput}
 [1] "2004-01-03 Eastern Standard Time" "2004-01-05 Eastern Standard Time"
 [3] "2004-01-17 Eastern Standard Time" "2004-01-19 Eastern Standard Time"
 [5] "2004-01-24 Eastern Standard Time" "2004-02-08 Eastern Standard Time"
 [7] "2004-02-12 Eastern Standard Time" "2004-02-13 Eastern Standard Time"
 [9] "2004-02-25 Eastern Standard Time" "2004-02-26 Eastern Standard Time"
\end{Soutput}
\end{Schunk}
The index scale of \code{z2} can be change to that of \code{z1} by
\begin{Schunk}
\begin{Sinput}
> index(z2) <- index(z1)
> z2
\end{Sinput}
\begin{Soutput}
Value:
 [1]  0.94306673 -0.04149429  0.59448077 -0.52575918 -0.96739776  0.95605566
 [7] -0.62733473 -0.92845336  0.56060280  0.08291711

Index:
 [1] "2004-01-05 Eastern Standard Time" "2004-01-14 Eastern Standard Time"
 [3] "2004-01-19 Eastern Standard Time" "2004-01-25 Eastern Standard Time"
 [5] "2004-01-27 Eastern Standard Time" "2004-02-07 Eastern Standard Time"
 [7] "2004-02-12 Eastern Standard Time" "2004-02-16 Eastern Standard Time"
 [9] "2004-02-20 Eastern Standard Time" "2004-02-24 Eastern Standard Time"
\end{Soutput}
\end{Schunk}

The start and the end of the index/time vector can be queried by
\code{start} and \code{end}:
\begin{Schunk}
\begin{Sinput}
> start(z1)
\end{Sinput}
\begin{Soutput}
[1] "2004-01-05 Eastern Standard Time"
\end{Soutput}
\begin{Sinput}
> end(z1)
\end{Sinput}
\begin{Soutput}
[1] "2004-02-24 Eastern Standard Time"
\end{Soutput}
\end{Schunk}


To work on both value and index/time, \pkg{zoo} provides method
a method to \code{window} and also adds a new generic 
\code{window<-} with a method for \code{"zoo"} objects. In both
cases the window is specified by
\begin{Scode}
window(x, index, start, end)
\end{Scode}
where \code{x} is the \code{"zoo"} object, \code{index} is a set
of indexes to be selected (by default the full index of \code{x})
and \code{start} and \code{end} can be used to restrict the 
\code{index} set. Thus, the first command in the following example
selects all observations starting from 2004--03--01 whereas the
second selects only from the observations with the 5th to 8th index
those up to 2004--03--01.
\begin{Schunk}
\begin{Sinput}
> window(Z, start = as.Date("2004-03-01"))
\end{Sinput}
\begin{Soutput}
           Aa         Bb         Cc        
2004-03-05 -1.2086102  1.4237978 -0.8161448
2004-03-10 -0.1103956  1.3477425  0.9552247
2004-03-14  0.8420238 -2.7384202  0.2315069
2004-03-20 -0.1901910  0.1230887 -1.5186216
\end{Soutput}
\begin{Sinput}
> window(Z, index = index(Z)[5:8], end = as.Date("2004-03-01"))
\end{Sinput}
\begin{Soutput}
           Aa          Bb          Cc         
2004-02-22  0.22542418  0.53838938  0.23136133
2004-02-29  1.20695518  0.31814222 -0.01129202
\end{Soutput}
\end{Schunk}
The same syntax can be used for the corresponding replacement function.
\begin{Schunk}
\begin{Sinput}
> window(z1, end = as.POSIXct("2004-02-01")) <- 9:5
> z1
\end{Sinput}
\begin{Soutput}
Value:
 [1]  9  8  7  6  5  6  7  8  9 10

Index:
 [1] "2004-01-05 Eastern Standard Time" "2004-01-14 Eastern Standard Time"
 [3] "2004-01-19 Eastern Standard Time" "2004-01-25 Eastern Standard Time"
 [5] "2004-01-27 Eastern Standard Time" "2004-02-07 Eastern Standard Time"
 [7] "2004-02-12 Eastern Standard Time" "2004-02-16 Eastern Standard Time"
 [9] "2004-02-20 Eastern Standard Time" "2004-02-24 Eastern Standard Time"
\end{Soutput}
\end{Schunk}

Two methods to generic functions that are standard in time series applications
are \code{lag} and \code{diff} which are available with the same
arguments as the \code{"ts"} methods---with the only exception that \code{diff}
not only allows for arithmetic but also geometric differences.
\begin{Schunk}
\begin{Sinput}
> lag(z1, k = -1)
\end{Sinput}
\begin{Soutput}
Value:
[1] 9 8 7 6 5 6 7 8 9

Index:
[1] "2004-01-14 Eastern Standard Time" "2004-01-19 Eastern Standard Time"
[3] "2004-01-25 Eastern Standard Time" "2004-01-27 Eastern Standard Time"
[5] "2004-02-07 Eastern Standard Time" "2004-02-12 Eastern Standard Time"
[7] "2004-02-16 Eastern Standard Time" "2004-02-20 Eastern Standard Time"
[9] "2004-02-24 Eastern Standard Time"
\end{Soutput}
\begin{Sinput}
> diff(z1)
\end{Sinput}
\begin{Soutput}
Value:
[1] -1 -1 -1 -1  1  1  1  1  1

Index:
[1] "2004-01-14 Eastern Standard Time" "2004-01-19 Eastern Standard Time"
[3] "2004-01-25 Eastern Standard Time" "2004-01-27 Eastern Standard Time"
[5] "2004-02-07 Eastern Standard Time" "2004-02-12 Eastern Standard Time"
[7] "2004-02-16 Eastern Standard Time" "2004-02-20 Eastern Standard Time"
[9] "2004-02-24 Eastern Standard Time"
\end{Soutput}
\end{Schunk}



\subsection[Coercion to and from "zoo"]{Coercion to and from \code{"zoo"}}
\label{sec:as.zoo}

Coercion to and from \code{"zoo"} objects is available for objects of
various classes, in particular \code{"ts"}, \code{"irts"} and \code{"its"}
objects can be coerced to \code{"zoo"} using the  respective \code{as.zoo}
method. The reverse coercion is available for \code{"its"} and for \code{"irts"}
(the latter in package \code{tseries}).
Furthermore, \code{"zoo"} objects can be coerced to vectors, matrices, lists and
data frames (the latter dropping the index/time attribute). A simple example is
\begin{Schunk}
\begin{Sinput}
> as.data.frame(Z)
\end{Sinput}
\begin{Soutput}
           Aa         Bb          Cc
1   1.2554339  0.6815732 -0.63292049
2  -1.4945833  1.3234122 -1.49442269
3  -1.8746225 -0.8732929  0.62733971
4  -0.1453861  0.4523490 -0.14597401
5   0.2254242  0.5383894  0.23136133
6   1.2069552  0.3181422 -0.01129202
7  -1.2086102  1.4237978 -0.81614483
8  -0.1103956  1.3477425  0.95522468
9   0.8420238 -2.7384202  0.23150695
10 -0.1901910  0.1230887 -1.51862157
\end{Soutput}
\end{Schunk}


\subsection[NA handling]{\code{NA} handling}

Three methods for dealing with \code{NA}s (missing observations) 
in the value are applicable to \code{"zoo"} objects:
\code{na.omit}, \code{na.contiguous}, \code{na.locf}.
\code{na.omit}---or its default method to be more precise---returns a \code{"zoo"}
object with incomplete observations removed. \code{na.contiguous}
extracts the longest consecutive stretch of non-missing values.
This function is made generic in \pkg{zoo} with a default method
(also applicable to \code{"zoo"} objects) and the \pkg{stats} function
being the \code{"ts"} method. Furthermore, a new generic function
\code{na.locf} and corresponding default method is introduced in \pkg{zoo}
which stands for \underline{l}ast \underline{o}bservation \underline{c}arried
\underline{f}orward. It replaces missing observations by the most recent
non-\code{NA} prior to it. Leading \code{NA}s, which cannot be replaced
by precious observations, are removed by default.

\begin{Schunk}
\begin{Sinput}
> z1[sample(1:10, 3)] <- NA
> z1
\end{Sinput}
\begin{Soutput}
Value:
 [1]  9 NA  7  6  5  6 NA  8  9 NA

Index:
 [1] "2004-01-05 Eastern Standard Time" "2004-01-14 Eastern Standard Time"
 [3] "2004-01-19 Eastern Standard Time" "2004-01-25 Eastern Standard Time"
 [5] "2004-01-27 Eastern Standard Time" "2004-02-07 Eastern Standard Time"
 [7] "2004-02-12 Eastern Standard Time" "2004-02-16 Eastern Standard Time"
 [9] "2004-02-20 Eastern Standard Time" "2004-02-24 Eastern Standard Time"
\end{Soutput}
\begin{Sinput}
> na.omit(z1)
\end{Sinput}
\begin{Soutput}
Value:
[1] 9 7 6 5 6 8 9
attr(,"na.action")
[1]  2  7 10
attr(,"class")
[1] "omit"

Index:
[1] "2004-01-05 Eastern Standard Time" "2004-01-19 Eastern Standard Time"
[3] "2004-01-25 Eastern Standard Time" "2004-01-27 Eastern Standard Time"
[5] "2004-02-07 Eastern Standard Time" "2004-02-16 Eastern Standard Time"
[7] "2004-02-20 Eastern Standard Time"
\end{Soutput}
\begin{Sinput}
> na.contiguous(z1)
\end{Sinput}
\begin{Soutput}
Value:
[1] 7 6 5 6
attr(,"na.action")
[1]  1  2  7  8  9 10
attr(,"class")
[1] "omit"

Index:
[1] "2004-01-19 Eastern Standard Time" "2004-01-25 Eastern Standard Time"
[3] "2004-01-27 Eastern Standard Time" "2004-02-07 Eastern Standard Time"
\end{Soutput}
\begin{Sinput}
> na.locf(z1)
\end{Sinput}
\begin{Soutput}
Value:
 [1] 9 9 7 6 5 6 6 8 9 9

Index:
 [1] "2004-01-05 Eastern Standard Time" "2004-01-14 Eastern Standard Time"
 [3] "2004-01-19 Eastern Standard Time" "2004-01-25 Eastern Standard Time"
 [5] "2004-01-27 Eastern Standard Time" "2004-02-07 Eastern Standard Time"
 [7] "2004-02-12 Eastern Standard Time" "2004-02-16 Eastern Standard Time"
 [9] "2004-02-20 Eastern Standard Time" "2004-02-24 Eastern Standard Time"
\end{Soutput}
\end{Schunk}



\section[Combining zoo with other packages]{Combining \pkg{zoo} with other packages}
\label{sec:combining}

The main purpose of the package \pkg{zoo} is to provide basic infrastructure for
working with indexed totally ordered observations that can be either employed by
useres directly or can be a basic ingredient on top of which other packages can
build. The latter is illustrated with a few brief examples involving the packages
\pkg{strucchange}, \pkg{tseries} and \pkg{fBasics} in this section.


\subsection[strucchange: empirical fluctuation processes of class "zoo"]{\pkg{strucchange}: empirical fluctuation processes of class \code{"zoo"}}

The package \pkg{strucchange} provides a collection of methods for testing,
monitoring and dating structural changes, in particular in linear regression models.
Tests for structural change assess whether the parameters of a model remain
constant over an ordering with respect to a specified variable, usually time.
To adequatly store and visualize empirical fluctuation processes which 
capture instabilities over this ordering, a data type for indexed ordered
observations is required. This was the motivation for starting the \pkg{zoo}
project.

A simple example for the need of \code{"zoo"} objects in \pkg{strucchange}
which could not (easily) be implemented by any of the other irregular time series classes
available on CRAN is described in the following. We assess the constancy of the
electrical resistance over the apparent juice content of kiwi fruits. The data
set \code{fruitohms} is contained in the \pkg{DAAG} package \citep{zoo:DAAG:2004}.
The fitted \code{ocus} object contains the OLS-based CUSUM process for the mean
of the electrical resistance (variable \code{ohms}) indexed by the juice
content (variable \code{juice}).

\begin{Schunk}
\begin{Sinput}
> library(strucchange)
> library(DAAG)
\end{Sinput}
\begin{Soutput}
Loading required package: leaps 
Loading required package: oz 
\end{Soutput}
\begin{Sinput}
> data(fruitohms)
> ocus <- gefp(ohms ~ 1, order.by = ~juice, data = fruitohms)
\end{Sinput}
\end{Schunk}

This OLS-based CUSUM process can be visualized using the \code{plot} method
for \code{"gefp"} objects which builds on the \code{"zoo"} method and yields in
this case the plot in Figure~\ref{fig:strucchange} showing the process which
crosses its 5\% critical value and 
thus signals a significant decrease in the mean electrical resistance over the
juice content. for more information on the package \pkg{strucchange} and the 
function \code{gefp} see \cite{zoo:Zeileis+Leisch+Hornik:2002} and 
\cite{zoo:Zeileis:2004}.

\begin{figure}
\begin{center}
\begin{Schunk}
\begin{Sinput}
> plot(ocus)
\end{Sinput}
\end{Schunk}
\includegraphics{zoo-strucchange2}
\caption{\label{fig:strucchange} Empirical M-fluctuation process for \code{fruitohms} data}
\end{center}
\end{figure}


\subsection[tseries: historical financial data]{\pkg{tseries}: historical financial data}

A typical application for irregular time series which became increasingly
important over the last years in computational statistics and finance is
daily (or higher frequent) financial data. The package \pkg{tseries} provides
the function \code{get.hist.quote} for obtaining historical financial data
by querying Yahoo! Finance at \url{http://finance.yahoo.com/},
an online portal quoting data provided by Reuters. The following code
queries the quotes of Lucent Technologies starting from 2001-01-01.

\begin{Schunk}
\begin{Sinput}
> library(tseries)
> LU <- get.hist.quote(instrument = "LU", start = "2001-01-01", 
+     end = "2004-09-30", origin = "1970-01-01")
\end{Sinput}
\end{Schunk}


In the returned \code{LU} object the irregular data is stored by extending
it to a regular grid and filling the gaps with \code{NA}s. The time is stored
in days starting from an \code{origin}, in this case specified to be 1970-01-01.
This series can be transformed easily into an irregula \code{"zoo"} series 
using a \code{"Date"} index. The log-difference returns for Lucent 
Technologies is depicted in Figure~\ref{fig:tseries}.

\begin{Schunk}
\begin{Sinput}
> LU <- zoo(value(LU), structure(time(LU), class = "Date"))
> LU <- na.omit(LU)
\end{Sinput}
\end{Schunk}

\begin{figure}
\begin{center}
\begin{Schunk}
\begin{Sinput}
> plot(diff(log(LU)))
\end{Sinput}
\end{Schunk}
\includegraphics{zoo-tseries3}
\caption{\label{fig:tseries} Log-difference returns for Lucent Technologies}
\end{center}
\end{figure}


\subsection[fBasics: indexes of class "timeDate"]{\pkg{fBasics}: indexes of class \code{"timeDate"}}

Although the methods in \pkg{zoo} work out of the box for many index classes,
it might be necessary for some index classes to provide \code{c}, \code{length},
\code{ORDER} and \code{MATCH} methods such that the methods in \pkg{zoo} 
work properly. An example for such an index class which requires a bit more
attention is \code{"timeDate"} from the \pkg{fBasics} package.

But after the necessary methods have been defined
\begin{Schunk}
\begin{Sinput}
> length.timeDate <- function(x) prod(x@Dim)
> ORDER.timeDate <- function(x, ...) order(as.POSIXct(x), ...)
> MATCH.timeDate <- function(x, y, ...) match(as.POSIXct(x), as.POSIXct(y), 
+     ...)
\end{Sinput}
\end{Schunk}
the class \code{"timeDate"} can be used for indexing \code{"zoo"} objects.
The following example illustrates how \code{z2} can be transformed
to use the \code{"timeDate"} class.
\begin{Schunk}
\begin{Sinput}
> library(fBasics)
\end{Sinput}
\begin{Soutput}
Rmetrics, (C) 1999-2004, Diethelm Wuertz, GPL
fBasics: Markets, Basic Statistics, Date and Time

Attaching package 'fBasics':


	The following object(s) are masked from package:zoo :

	 as.POSIXct.timeDate as.timeSeries 
\end{Soutput}
\begin{Sinput}
> z2td <- zoo(value(z2), timeDate(index(z2), FinCenter = "GMT"))
> z2td
\end{Sinput}
\begin{Soutput}
Value:
 [1]  0.94306673 -0.04149429  0.59448077 -0.52575918 -0.96739776  0.95605566
 [7] -0.62733473 -0.92845336  0.56060280  0.08291711

Index:
[1] "GMT"
 [1] [2004-01-05] [2004-01-14] [2004-01-19] [2004-01-25] [2004-01-27]
 [6] [2004-02-07] [2004-02-12] [2004-02-16] [2004-02-20] [2004-02-24]
\end{Soutput}
\end{Schunk}

\section{Summary} \label{sec:summary}

The package \pkg{zoo} provides an \proglang{S3} class and methods
for indexed totally ordered observations, such as irregular time series.
its key design goals are independence of a particular index class 
and compatibility with standard generics similar to the behaviour of 
the corresponding \code{"ts"} methods. This paper describes how
these are implemented in \pkg{zoo} and illustrates the usage of 
the methods for plotting, merging and
binding, several mathematical operations, extracting and replacing value
and index, coercion and \code{NA} handling.

\bibliography{zoo}

\begin{appendix}
\begin{tabular}{rp{11cm}}
\multicolumn{2}{l}{\textbf{Creation}} \\
\code{zoo(x, order.by)} & creation of a \code{"zoo"} object
  from the observations \code{x} (a vector or a matrix) and an index
  \code{order.by} by which the observations are ordered. \\
& For computations on arbitrary index classes, methods to the 
  following genric functions are assumed to work: combining \code{c()},
  querying lenght \code{length()}, subsetting \code{[,}, ordering
  \code{ORDER()} and value matching \code{MATCH()}.\\[0.5cm]

\multicolumn{2}{l}{\textbf{Standard methods}} \\
\code{plot} & plotting \\
\code{lines} & adding a \code{"zoo"} series to a plot \\
\code{print} & printing \\
\code{summary} & summarizing (column-wise) \\
\code{str} & displaying structure of \code{"zoo"} objects \\
\code{head, tail} & head and tail of \code{"zoo"} objects \\[0.5cm]

\multicolumn{2}{l}{\textbf{Coercion}} \\
\code{as.zoo} & coercion to \code{"zoo"} is available for objects
    of class \code{"ts"}, \code{"its"}, \code{"irts"} (plus a default
    method).\\
\code{as.}\textit{class}\code{.zoo} & coercion from \code{"zoo"} to
    other classes. Currently available \textit{class} in \code{"matrix"},
    \code{"vector"}, \code{"data.frame"}, \code{"list"}, \code{"irts"}
    and \code{"its"}. \\
\code{is.zoo} & querying wether ab object is of class \code{"zoo"} \\[0.5cm]

\multicolumn{2}{l}{\textbf{Merging and binding}} \\
\code{merge} & union, intersection, left join, right join along indexes\\
\code{cbind} & column binding along the intersection of the index\\
\code{rbind} & row binding (indexes may not overlap)\\
\code{aggregate} & compute summary statistics along a coarser grid of indexes \\[0.5cm]

\multicolumn{2}{l}{\textbf{Mathematical operations}} \\
\code{Ops} & group generic functions performed along the intersection of indexes\\
\code{t} & transposing (coerces to \code{"matrix"} before \\
\code{cumsum} & compute (columnwise) cumulative quantities: sums
    \code{cumsum()}, products \code{cumprod()}, maximum \code{cummax()},
    minimum \code{cummin()}.\\[0.5cm]

\multicolumn{2}{l}{\textbf{Extracting and replacing data and index}} \\
\code{index, time} & extract the index of a series\\
\code{index<-, time<-} & replace the index of a series\\
\code{value} & extract the data associated with a \code{"zoo"} object
    (change to \code{coredata}?)\\
\code{value<-} & replace data\\    
\code{lag} & lagged observations \\
\code{diff} & arithmetic and geometric differences \\
\code{start, end} & querying start and end of a series \\
\code{window, window<-} & subsetting of \code{"zoo"} objects
    using their index\\[0.5cm]

\multicolumn{2}{l}{\textbf{\code{NA} handling}} \\
\code{na.omit} & omit \code{NA}s \\
\code{na.contiguous} & compute longest sequence of non-\code{NA} observations \\
\code{na.locf} & eliminate \code{NA}s by carrying forward the last observation\\
\code{na.approx} & eliminate \code{NA}s by interpolation\\[0.5cm]

%% \multicolumn{2}{l}{\textbf{Rolling operations}} \\
%% \code{runmean} & running mean, median and maxim are \code{runmean}, \code{runmed} and
%%   \code{runmax}, respectively
\end{tabular}

\end{appendix}

\end{document}
